%--------------------

%#####################################
\section{函数模板}
%#####################################



%-------------------------------------

\begin{frame}[fragile]{7.1~函数模板}
\vspace{-4mm}
\begin{block}{泛型编程}
\begin{itemize}
  \item 一种采用与数据类型无关的方式编写代码的方法,是\alert{代码重用}的重要手段。\onslide<2->{如何设计一个通用的排序算法?}
  \item <3->\alert{模板}是泛型编程的基础,它将数据类型参数化,为数据结构和算法的\alert{抽象}提供\alert{通用的代码}解决方案
\end{itemize}

\end{block}

\uncover<3->{请观察下面两组代码:}

\vspace{-4mm}

\begin{columns}[t]

\column{0.65\textwidth}
\begin{blueblock}<3->{\texttt{getMax}函数定义一}
\vspace{-1.5mm}\begin{lstlisting}
const int& getMax(const int &a, const int &b){
    return a>b ? a : b;
}
\end{lstlisting}\vspace{-1.5mm}
\end{blueblock}
\vspace{-2mm}
\begin{blueblock}<3->{\texttt{getMax}函数定义二}
\vspace{-1.5mm}\begin{lstlisting}
const string& getMax(const string &a, const string &b){
    return a>b ? a : b;
}
\end{lstlisting}\vspace{-1.5mm}
\end{blueblock}

\column{0.3\textwidth}
\begin{greenblock}<4->{问题}
\begin{itemize}
  \item 两个函数定义有什么异同?
  \item <5-> 有什么办法可以简化?
\end{itemize}

\end{greenblock}

\end{columns}

\end{frame}

%-------------------------------------
\subsection{定义函数模板}
%-------------------------------------

\begin{frame}[fragile]{7.1.1~定义函数模板}
\vspace{-3mm}
函数模板实现\alert{一类函数}的\alert{通用}代码解决方案,即实现某种功能的\alert{一类函数的抽象}:

\vspace{-4mm}

\begin{columns}[t]

\column{0.65\textwidth}
\begin{blueblock}{\texttt{getMax}函数模板定义}

\begin{lstlisting}[moreemph={T}]
const int& getMax(const int &a, const int &b){
    return a>b ? a : b;
}
\end{lstlisting}
\begin{tikzpicture}[overlay]
\node at (7., 0.3) [rectangle, draw=blue!50, fill=blue!20, text=blue!70] {类型参数化};
\draw[->,blue,line width =1pt] (1.4,1.4) -- (3.1,-0.4);
\draw[->,blue,line width =1pt] (4.3,1.4) -- (3.2,-0.4);
\draw[->,blue,line width =1pt] (6.5,1.4) -- (3.3,-0.4);
\end{tikzpicture}
\begin{lstlisting}[moreemph={T}]
template <typename T>
const T& getMax(const T &a, const T &b){
    return a>b ? a : b;
}
\end{lstlisting}

\end{blueblock}

\begin{redblock}<3->{注意}
\begin{itemize}
  \item 注意不要混淆\alert{模板参数}和\alert{函数形参}的概念
  \item 模板的声明和定义应放在同一个头文件里
\end{itemize}
\end{redblock}

\column{0.3\textwidth}
\begin{yellowblock}<2->{说明}
$\bullet$ 模板的定义以关键字~\alert{\texttt{template}}~开始\\
$\bullet$ 模板参数列表放在一对\alert{尖括号}里面\\
$\bullet$ 每一个参数前面用\\\alert{\texttt{typename}}~或~\alert{\texttt{class}}~声明\\
$\bullet$ 列表含有多个模板参数则参数之间用\alert{逗号}分开
\end{yellowblock}

\end{columns}
\end{frame}

%-------------------------------------
\subsection{实例化函数模板}
%-------------------------------------

\begin{frame}[fragile]{7.1.2~实例化函数模板\small{~---~模板参数推断}}

如何使用函数模板? \onslide<2->{实例化模板函数需要提供具体的\alert{数据类型或值}}

\vspace{-4mm}

\begin{columns}[t]

\column{0.75\textwidth}
\begin{blueblock}<3->{实例化方法一:隐式推断类型}
\begin{lstlisting}
cout << getMax(1.0, 2.5) << endl; // T被推断为double
\end{lstlisting}
生成如下函数实例
\begin{lstlisting}
const double& getMax(const double &a, const double &b){
    return a>b ? a : b;
}
\end{lstlisting}

\end{blueblock}
\begin{blueblock}<4->{实例化方法二:显式指定类型}
\begin{lstlisting}
cout << getMax<double>(1.0, 2.5) << endl;      //显式指明T为double
cout << getMax<string>("Hi ", "C++") << endl;  //显式指明T为string
\end{lstlisting}
\end{blueblock}

\column{0.2\textwidth}
\begin{yellowblock}<3->{说明}
编译器在\alert{编译}的过程中,利用实参来推断模板参数的类型
\end{yellowblock}
\vspace{0.7cm}
\begin{yellowblock}<4->{说明}
用户显式地指明模板参数的类型
\end{yellowblock}

\end{columns}

\end{frame}

%-----------------

\begin{frame}[fragile]{7.1.2~实例化函数模板\normalsize{~---~为类类型添加模板支持}}

当模板函数的实参为\alert{类类型}时,需要为类对象添加模板使用到的相关操作:

\vspace{-4mm}

\begin{columns}[t]
\column{0.65\textwidth}
\begin{blueblock}<2->{示例代码}
\begin{lstlisting}[moreemph={Fraction}]
Fraction a(3,4),b(2,5);
cout << getMax(a, b) << endl; // T为Fraction类型
\end{lstlisting}
\end{blueblock}

\begin{blueblock}<3->{函数模板实例化}
\begin{lstlisting}
const Fraction& getMax(const Fraction &a, const Fraction &b){
    return a>b ? a : b;
}
\end{lstlisting}
\end{blueblock}

\column{0.3\textwidth}
\begin{greenblock}<4->{问题}
在编译上面代码时提示编译错误,原因可能是什么?
\end{greenblock}
\begin{greenblock}<5->{答案}
在\texttt{getMax}模板内部用到了关系\texttt{>}运算,但\texttt{Fraction}类不支持关系\texttt{>}运算
\end{greenblock}
\end{columns}

\end{frame}

%-----------------

\begin{frame}[fragile]{7.1.2~实例化函数模板\normalsize{~---~为类类型添加模板支持}}

给\texttt{Fraction}类型添加关系\texttt{>}运算支持:

\vspace{-4mm}

\begin{columns}[t]
\column{0.75\textwidth}
\begin{blueblock}{\texttt{Fraction}类~关系\texttt{>}运算~声明及定义}
\begin{lstlisting}[moreemph={Fraction}]
class Fraction{
    // 将关系>运算声明为Fraction类的友元
    friend bool operator>(const Fraction &lhs, const Fraction &rhs);
    // 其它成员与之前一致
    ...
};

bool operator>(const Fraction &lhs, const Fraction &rhs){
    return lhs.m_numerator*rhs.m_denominator > lhs.m_denominator*rhs.m_numerator;
}
\end{lstlisting}
\end{blueblock}
\column{0.2\textwidth}
\begin{yellowblock}{说明}
根据运算符重载的原则将关系运算符函数\texttt{operator>}作为\texttt{Fraction}类的辅助函数,并将其声明为\texttt{Fraction}类的友元
\end{yellowblock}
\end{columns}

\end{frame}

%-------------------------------------
\subsection{模板参数类型}
%-------------------------------------

\begin{frame}[fragile]{7.1.3~模板参数类型}

以下两组代码中,函数\alert{模板参数}有什么异同?

\vspace{-4mm}

\begin{columns}[t]
\column{0.5\textwidth}
\begin{blueblock}{\texttt{foo}函数定义}
\begin{tikzpicture}[overlay]
  \visible<2->{\draw[red,thick] (3.7,-0.3)--(5.5,-0.3);}
\end{tikzpicture}
\vspace{-4mm}
\begin{lstlisting}[moreemph={T,U}]
template <typename T, typename U>
T foo(const T &t, const U &u) {
    return T(t);
}
\end{lstlisting}
\end{blueblock}
\begin{blueblock}{\texttt{maxElem}函数定义}
\begin{tikzpicture}[overlay]
  \visible<2->{\draw[red,thick] (3.5,-0.3)--(5.0,-0.3);}
\end{tikzpicture}
\vspace{-4mm}
\begin{lstlisting}[moreemph={T}]
template<typename T, int size>
const T& maxElem(T (&arr)[size]) {
    T *p = &arr[0];
    for (auto i = 0; i < size; ++i)
        if (*p < arr[i])
            p = &arr[i];
    return *p;
}
\end{lstlisting}
\end{blueblock}


\column{0.45\textwidth}
\begin{block}<2->{类型参数}
作为\alert{类型说明符},指定函数的返回值类型、形参类型以及函数体内对象的类型等
\end{block}
\begin{block}<2->{非类型参数}
代表一个值,当编译器实例化该模板时必须要为其提供一个\alert{常量}表达式
\end{block}
\begin{yellowblock}<3->{说明}
\texttt{maxElem}函数模板中的函数形参\texttt{arr}为一个指向含有\texttt{size}个\texttt{T}类型数据元素数组的引用
\end{yellowblock}
\end{columns}

\end{frame}

%-------------------

\begin{frame}[fragile]{7.1.3~模板参数类型}

调用\texttt{maxElem}函数:

\vspace{-4mm}

\begin{columns}[t]

\column{0.65\textwidth}
\begin{blueblock}{\texttt{maxElem}函数模板实例化}
\begin{lstlisting}
int arr[10] = {1,8,5,3};
int x = maxElem(arr);
// 或者显式调用 maxElem<int,10>(arr);
\end{lstlisting}
\end{blueblock}
编译器将会生成如下版本的函数:
\begin{blueblock}{}
\begin{lstlisting}
const int& maxElem(int (&arr)[10]);
\end{lstlisting}
\end{blueblock}

\column{0.3\textwidth}
\begin{greenblock}<2->{问题}
传递数组参数还有什么方式?
\end{greenblock}
\begin{greenblock}<3->{答案}
还可以通过指针传递数组首地址的方式
\end{greenblock}

\end{columns}

\end{frame}

%------------------

\begin{frame}[fragile]{7.1.3~模板参数类型\normalsize{~---~模板重载与特化}}

如果前面定义的\texttt{getMax}函数模板在调用过程中的实参为指针类型:

\vspace{-4mm}

\begin{columns}[t]

\column{0.65\textwidth}
\begin{blueblock}{\texttt{getMax}函数调用一}
\begin{lstlisting}[moreemph={T}]
int a = 1, b = 2;
getMax(&a, &b);
\end{lstlisting}
\end{blueblock}
\begin{blueblock}{\texttt{getMax}定义一}
\begin{lstlisting}[moreemph={T}]
template <typename T>
const T& getMax(const T &a, const T &b){
    return a > b ? a : b;
}
\end{lstlisting}
\begin{tikzpicture}[overlay]
  \visible<3->{\draw[blue,->,line width=1.5pt] (2,0.5)--(2,0);}
\end{tikzpicture}
\begin{onlyenv}<3->
\begin{lstlisting}[moreemph={T}]
const int* & getMax(const int* &a, const int* &b){
    return a > b ? a : b;
}
\end{lstlisting}
\end{onlyenv}
\end{blueblock}

\column{0.3\textwidth}
\begin{yellowblock}{说明}
需要返回两个指针所指向的对象的比较结果
\end{yellowblock}
\begin{greenblock}{问题}
\texttt{getMax}函数模板的定义还能否满足要求?
\end{greenblock}
\begin{greenblock}<2->{答案}
不能。编译器推演出的参数\texttt{T}为\texttt{int*},函数体里面的操作变成了两个指针对象的比较
\end{greenblock}

\end{columns}

\end{frame}

%-----------------

\begin{frame}[fragile]{7.1.3~模板参数类型\normalsize{~---~模板重载与特化}}

为此,需要\alert{重载}一个\texttt{getMax}模板函数:

\vspace{-4mm}

\begin{columns}[t]

\column{0.65\textwidth}
\begin{blueblock}{\texttt{getMax}函数调用二}
\begin{lstlisting}[moreemph={T}]
int a = 1, b = 2;
getMax(&a, &b);
\end{lstlisting}
\end{blueblock}
\begin{blueblock}{\texttt{getMax}函数模板重载}
\begin{lstlisting}[moreemph={T}]
template <typename T>
T* const & getMax( T* const &a, T* const &b){
    return *a>*b ? a : b;
}
\end{lstlisting}
\end{blueblock}

\column{0.3\textwidth}
\begin{yellowblock}{说明}
模板实参\texttt{T}的类型为\texttt{int},\texttt{*a}和\texttt{*b}指向的是\texttt{int}对象,函数体里面的操作是两个\texttt{int}对象的比较
\end{yellowblock}

\end{columns}

\end{frame}

%-----------------

\begin{frame}[fragile]{7.1.3~模板参数类型\normalsize{~---~模板重载与特化}}

进一步,如果调用的实参是指向字符串常量的指针:

\vspace{-4mm}

\begin{columns}[t]

\column{0.65\textwidth}
\begin{blueblock}{\texttt{getMax}函数调用三}
\begin{lstlisting}[moreemph={T}]
const char *a = "Hi", *b = "C++";
cout << getMax(a, b) << endl;
\end{lstlisting}
\end{blueblock}
\begin{blueblock}{\texttt{getMax}函数定义二}
\vspace{-2.5mm}\begin{lstlisting}[moreemph={T}]
template <typename T>
const T & getMax(const T &a, const T &b){
    return a > b ? a : b;
}

template <typename T>
T* const & getMax( T* const &a, T* const &b){
    return *a>*b ? a : b;
}
\end{lstlisting}\vspace{-2mm}
\end{blueblock}

\column{0.3\textwidth}
\begin{yellowblock}{说明}
需要返回指向字符串值较大的字符指针
\end{yellowblock}
\begin{greenblock}<2->{问题}
现有的两个\texttt{getMax}函数的定义还能否满足要求?
\end{greenblock}
\begin{greenblock}<3->{答案}
不能。\texttt{*a}和\texttt{*b}指向的是单个字符,函数体里面的操作变成了两个字符的比较
\end{greenblock}

\end{columns}

\end{frame}

%-----------------

\begin{frame}[fragile]{7.1.3~模板参数类型\normalsize{~---~模板重载与特化}}

为此,需要\alert{特例化}一个\texttt{getMax}模板函数:

\vspace{-4mm}

\begin{columns}[t]

\column{0.65\textwidth}
\begin{blueblock}{\texttt{getMax}函数调用三}
\begin{lstlisting}[moreemph={T}]
const char *a = "Hi", *b = "C++";
cout << getMax(a, b) << endl;
\end{lstlisting}
\end{blueblock}
\begin{blueblock}{\texttt{getMax}函数模板特化}
\begin{lstlisting}[moreemph={T}]
template <>
const char* const & getMax(const char* const &a, const char* const &b){
    return strcmp(a,b) > 0 ? a : b;
}
\end{lstlisting}
\end{blueblock}


\column{0.3\textwidth}
\begin{yellowblock}{说明}
模板参数列表为空,表明将显式提供所有模板实参
\end{yellowblock}
\vspace{-2mm}
\begin{yellowblock}{说明}
\texttt{T}被推断为\texttt{const char*},\texttt{a}和\texttt{b}分别为一个指向\texttt{const char}对象的\texttt{const}指针的引用,函数是对两个字符串值的比较
\end{yellowblock}
\vspace{-2mm}
\begin{greenblock}<2->{问题}
为什么会选择特例化的版本?
\end{greenblock}

\end{columns}

\end{frame}

%-----------------

\begin{frame}[fragile]{7.1.3~模板参数类型\normalsize{~---~模板重载与特化}}

还可以通过模板特化改善算法:

\vspace{-4mm}

\begin{columns}[t]

\column{0.65\textwidth}
\begin{blueblock}<2->{\texttt{Swap}函数模板定义}
\vspace{-2.5mm}\begin{lstlisting}[moreemph={T}]
template<typename T>
void Swap(T &a, T &b) {
    T c(a); // 复制构造对象 c
    a = b;
    b = c;
}
\end{lstlisting}\vspace{-2mm}
\end{blueblock}

\column{0.3\textwidth}
\begin{yellowblock}<2->{说明}
需要构造一个辅助的局部对象\texttt{c},才能完成\texttt{a}和\texttt{b}的交换
\end{yellowblock}

\end{columns}


\uncover<3->{如果\texttt{T}是\texttt{int},可以利用模板特化做出优化:}

\vspace{-4mm}

\begin{columns}[t]

\column{0.65\textwidth}
\begin{blueblock}<4->{\texttt{Swap}函数模板特化}
\vspace{-2.5mm}\begin{lstlisting}[moreemph={T}]
template<>
void Swap(int &a, int &b)
    a ^= b;
    b ^= a;
    a ^= b;
}
\end{lstlisting}\vspace{-2mm}
\end{blueblock}

\column{0.3\textwidth}
\begin{yellowblock}<4->{说明}
利用异或操作完成两个整数的交换,没有创建辅助对象,没有产生构造和析构行为,提高了执行效率
\end{yellowblock}

\end{columns}

\end{frame}

%-----------------

%-------------------------------------
\subsection{类成员模板}
%-------------------------------------

%-----------------
\begin{frame}[fragile]{7.1.4~类成员模板}

类的成员函数也可以定义为函数模板:

\vspace{-4mm}

\begin{columns}[t]

\column{0.6\textwidth}
\begin{blueblock}{类\texttt{X}定义}
\begin{lstlisting}[moreemph={T}]
class X{
    void * m_p = nullptr;
public:
    template <typename T>
    void reset(T *t) { m_p = t; }
};
\end{lstlisting}
\end{blueblock}
\begin{blueblock}<2->{\texttt{reset}函数调用}
\begin{lstlisting}[moreemph={T}]
int i = 0;
double d = 0;
X x;
x.reset(&i); // 或者x.reset<int>(&i);
x.reset(&d); // 或者x.reset<double>(&d);
\end{lstlisting}
\end{blueblock}

\column{0.35\textwidth}
\begin{yellowblock}{说明}
成员函数\texttt{reset}定义为一个函数模板,接受不同类型的指针实参
\end{yellowblock}
\begin{yellowblock}<2->{说明}
$\bullet$ 第一条\texttt{reset}函数调用中\texttt{T}被推断为\texttt{int}类型,\texttt{m\_p}存放整型对象\texttt{i}的地址\\
$\bullet$ 第二条\texttt{reset}函数调用中\texttt{T}被推断为\texttt{double}类型,\texttt{m\_p}存放\texttt{double}类对象\texttt{d}的地址
\end{yellowblock}

\end{columns}

\end{frame}
%-----------------



%-------------------------------------
\subsection{可变参函数模板}
%-------------------------------------

\begin{frame}[fragile]{7.1.5~可变参函数模板}

C++11新标准允许我们使用\alert{数目可变}的模板参数:

\vspace{-4mm}

\begin{columns}[t]

\column{0.65\textwidth}
\begin{blueblock}{\texttt{foo}函数定义}
\begin{lstlisting}[moreemph={T}]
template<typename... Args >
void foo(Args... args) {
    // 打印参数包args中参数的个数
    cout << sizeof...(args) << endl;
}
\end{lstlisting}
\end{blueblock}
\begin{blueblock}<2->{\texttt{foo}函数调用}
\begin{lstlisting}[moreemph={T}]
foo();                  // 输出:0
foo(1,1.5);             // 输出:2
foo(1,1.5,"C++");       // 输出:3
\end{lstlisting}
\end{blueblock}

\column{0.3\textwidth}
\begin{yellowblock}{说明}
$\bullet$ 可变数目的参数称为\alert{参数包},用省略号“\alert{\texttt{...}}”表示,可包含0到任意个模板参数\\
$\bullet$ \texttt{foo}函数的形参\texttt{args}为模板参数包类型,接受可变数目的实参
\end{yellowblock}
\end{columns}
\end{frame}

%-----------------

\begin{frame}[fragile]{7.1.5~可变参函数模板\normalsize{~---~包展开}}

可以通过\alert{递归}的方式展开函数模板参数包:

\vspace{-4mm}

\begin{columns}[t]

\column{0.68\textwidth}
\begin{blueblock}{\texttt{print}函数定义}
\vspace{-2mm}\begin{lstlisting}[moreemph={T}]
template<typename T, typename... Args>
ostream& print(ostream &os, const T &t, const Args&... rest) {
    os << t << " "; // 打印第一个参数
    return print(os, rest...); // 递归调用
}

template<typename T>
ostream& print(ostream &os, const T &t) {
    return os << t; // 打印最后一个参数
}
\end{lstlisting}\vspace{-2mm}
\end{blueblock}
\begin{blueblock}<2->{\texttt{print}函数调用}
\vspace{-1.5mm}\begin{lstlisting}[moreemph={T}]
print(cout,1, 2.5, "C++") << endl; // 输出1 2.5 C++
\end{lstlisting}\vspace{-1.5mm}
\end{blueblock}

\column{0.27\textwidth}
\begin{yellowblock}{说明}
$\bullet$ 第一次处理参数包中的第一个参数,然后用剩余参数调用自身\\
$\bullet$ 当参数包里面只剩下一个参数时,非可变参模板与可变参模板都匹配,但是非可变参模板更特例化,编译器首选非可变参数版本
\end{yellowblock}

\end{columns}

\end{frame}

%-----------------

\begin{frame}[fragile]{7.1.5~可变参函数模板\normalsize{~---~转发参数包}}
\begin{block}{在泛型编程中,常常需要将参数原封不动的转发给另外一个函数}
\begin{tikzpicture}[overlay]
\visible<2->{\draw[color=red!70,->,line width=1.5pt] (6.5,-2.5) -- node [color=red!70,pos=0.45,above,sloped]{改进效率}(8,-2.5);}
\end{tikzpicture}
\begin{columns}
\column{0.4\textwidth}
\begin{lstlisting}[moreemph={T}]
template<typename TYPE, typename ARG>
TYPE* get_instance(const ARG arg){

     TYPE *ret;
     ret = new TYPE(arg);
     return ret;
}
\end{lstlisting}
\column{0.4\textwidth}
\begin{onlyenv}<2->
\begin{lstlisting}[moreemph={T}]
template<typename TYPE, typename ARG>
TYPE* get_instance(const ARG  (*@\alert{\&}@*)arg){

     TYPE *ret;
     ret = new TYPE(arg);
     return ret;
}
\end{lstlisting}
\end{onlyenv}
\end{columns}
\end{block}

\begin{greenblock}<3->{问题}
如果形参类型是右值,如何处理?
\end{greenblock}
\end{frame}
\begin{frame}[fragile]{7.1.5~可变参函数模板\normalsize{~---~转发参数包}}

两个函数的形参都是\alert{右值引用},\texttt{forwardValue}函数调用报错,为什么?

\vspace{-4mm}

\begin{columns}[t]

\column{0.65\textwidth}
\begin{blueblock}{\texttt{forwardValue}函数及\texttt{rvalue}函数定义}
\vspace{-2mm}\begin{lstlisting}[moreemph={T}]
void rvalue(int &&val){}

template<typename T>
void forwardValue(T &&val) {
    rvalue(val);
}
\end{lstlisting}\vspace{-2mm}
\end{blueblock}
\begin{blueblock}{\texttt{forwardValue}函数调用}
\vspace{-2mm}\begin{lstlisting}[moreemph={T}]
forwardValue(42);   // 错误
\end{lstlisting}\vspace{-2mm}
\end{blueblock}
\begin{blueblock}<2->{\texttt{forwardValue}函数调用细节}
\vspace{-2mm}\begin{lstlisting}[moreemph={T}]
T &&val = 42;       //等同于auto &&val = 42
rvalue(val);
\end{lstlisting}\vspace{-2mm}
\end{blueblock}

\column{0.3\textwidth}
\begin{greenblock}<2->{答案}
$\bullet$ 右值引用声明 \texttt{\&\&}与类型推导结合可以与右值绑定,所以\texttt{val}为右值引用\\
$\bullet$ 右值引用\texttt{val}引用右值42,但\alert{\texttt{val}本身是左值}\\
$\bullet$ \texttt{rvalue}函数只接受右值形参,但\texttt{val}是左值
\end{greenblock}
\begin{greenblock}<3->{问题}
可以让传入\texttt{rvalue}函数的\alert{实参保持原属性吗}?
\end{greenblock}

\end{columns}

\end{frame}
%-----------------

\begin{frame}[fragile]{7.1.5~可变参函数模板\normalsize{~---~转发参数包}}

在C++11新标准下可以利用~\alert{\texttt{std::forward}}~函数实现:

\vspace{-4mm}

\begin{columns}[t]

\column{0.6\textwidth}
\begin{blueblock}{\texttt{std::forward}转发左值描述性声明}
\vspace{-2mm}\begin{lstlisting}[basicstyle=\scriptsize\ttfamily,moreemph={T}]
template<typename T>
T&& forward( typename std::remove_reference<T>::type& t );
\end{lstlisting}\vspace{-2mm}
\end{blueblock}
\begin{blueblock}<2->{\texttt{forwardValue}函数定义二}
\vspace{-2mm}
\begin{lstlisting}[moreemph={T}]
template<typename T>
void forwardValue(T &&val) {
    rvalue(std::forward<T>(val));
}
\end{lstlisting}
\end{blueblock}
\begin{blueblock}<2->{\texttt{forwardValue}函数调用二}
\vspace{-2mm}
\begin{lstlisting}[moreemph={T}]
forwardValue(42);       //正确
int a = 42;
forwardValue(a);        //正确
\end{lstlisting}
\end{blueblock}

\column{0.35\textwidth}
\begin{yellowblock}<3->{说明}
$\bullet$ 当传入\texttt{forwardValue}的实参为右值,\texttt{T}被推断为非引用类型,forward<T>返回右值引用\\
$\bullet$ 当传入\texttt{forwardValue}的实参为左值,\texttt{T}被推断为左值引用类型,此时\texttt{forward<T>}将返回左值引用
\end{yellowblock}
\vspace{-2mm}
\begin{redblock}<4->{注意}
在C++11新标准下,\\
\texttt{\&\& \&和\& \&\&}折叠为 \texttt{\&}
\end{redblock}
\end{columns}
\end{frame}

%###############################################################################
\section{类模板}
%###############################################################################

%-----------------

\begin{frame}[fragile]{7.2~类模板}

类似函数模板,可以定义一个类模板用来生成具有\alert{相同结构}的一族类实例:

\vspace{-4mm}

\begin{columns}[t]

\column{0.65\textwidth}
\begin{blueblock}{\texttt{Array}类模板定义}
\begin{lstlisting}[moreemph={T}]
template<typename T, size_t N>
class Array {
    T m_ele[N];
public:
    Array() {}
    Array(const std::initializer_list<T> &);
    T& operator[](size_t i);
    constexpr size_t size() { return N; }
};
\end{lstlisting}
\end{blueblock}

\column{0.3\textwidth}
\begin{yellowblock}{说明}
$\bullet$ 类型参数\texttt{T}和非类型参数\texttt{N},分别用来表示元素的类型和元素的数目\\
$\bullet$ \alert{\texttt{initializer\_list}}~类型是C++11标准库提供的新类型,支持具有相同类型但数量未知的列表类型
\end{yellowblock}

\end{columns}

\end{frame}

%-----------------

%-------------------------------------
\subsection{成员函数定义}
%-------------------------------------

%-----------------

\begin{frame}[fragile]{7.2.1~成员函数定义}

与普通类相同,既可以在类的内部,也可以在\alert{类的外部}定义\alert{类模板成员函数}:

\vspace{-4mm}

\begin{columns}[t]

\column{0.65\textwidth}
\begin{blueblock}{\texttt{Array}类模板~构造函数~类外定义}
\begin{lstlisting}[moreemph={T,Array,initializer}]
template<typename T, size_t N>
Array<T, N>::Array(const std::initializer_list<T> &l):m_ele{T()}{
    size_t m = l.size() < N ? l.size() : N;
    for (size_t i = 0; i < m; ++i) {
        m_ele[i] = *(l.begin() + i);
    }
}
\end{lstlisting}
\end{blueblock}

\column{0.3\textwidth}
\begin{yellowblock}{说明}
$\bullet$ 必须以关键字\texttt{template}开始,后接\alert{与类模板相同}的模板参数列表\\
$\bullet$ 紧随类名后面的参数列表代表一个实例化的实参列表,每个参数不需要\texttt{typename}或\texttt{class}说明符
\end{yellowblock}

\end{columns}

\end{frame}

%-----------------

\begin{frame}[fragile]{7.2.1~成员函数定义}

与普通类相同,既可以在类的内部,也可以在\alert{类的外部}定义\alert{类模板成员函数}:

\vspace{-4mm}

\begin{columns}[t]

\column{0.65\textwidth}
\begin{blueblock}{\texttt{Array}类模板~构造函数~类外定义}
\vspace{-2mm}
\begin{lstlisting}[moreemph={T,Array,initializer}]
template<typename T, size_t N>
Array<T, N>::Array(const std::initializer_list<T> &l):m_ele{T()}{
    size_t m = l.size() < N ? l.size() : N;
    for (size_t i = 0; i < m; ++i) {
        m_ele[i] = *(l.begin() + i);
    }
}
\end{lstlisting}
\end{blueblock}
\begin{blueblock}<2->{\texttt{Array}类模板~\texttt{[]}运算符函数~类外定义}
\vspace{-2mm}
\begin{lstlisting}[moreemph={T,Array}]
template<typename T, size_t N>
T& Array<T, N>::operator[](size_t i) {
    return m_ele[i];
}
\end{lstlisting}
\end{blueblock}

\column{0.3\textwidth}
\begin{yellowblock}{说明}
$\bullet$ \texttt{m\_ele}中的每一个元素用T类型的默认初始化方式初始化\\
$\bullet$ 将形参\texttt{l}中的元素依次复制\\
$\bullet$ \texttt{l.begin}返回列表\texttt{l}中第一个元素的迭代器
\end{yellowblock}
\begin{yellowblock}<2->{说明}
类模板的下标运算符函数返回数组\texttt{m\_ele}中第\texttt{i}个元素的引用
\end{yellowblock}

\end{columns}

\end{frame}

%-----------------

%-------------------------------------
\subsection{实例化类模板}
%-------------------------------------

%-----------------

\begin{frame}[fragile]{7.2.2~实例化类模板}

当使用一个类模板时,我们需要\alert{显式}提供模板参数信息,即\alert{模板实参列表}:

\vspace{-4mm}

\begin{columns}[t]

\column{0.65\textwidth}
\begin{blueblock}{实例化\texttt{Array}类模板}
\begin{lstlisting}[moreemph={Array}]
Array<char, 5> a; //创建一个Array<char, 5>类型对象 a
Array<int, 5> b = {1,2,3}; //创建一个Array<int, 5>类型对象 b
\end{lstlisting}
\end{blueblock}
\uncover<2->{下面代码逐个输出对象\texttt{b}的每一个元素:}
\begin{blueblock}<2->{}
\begin{lstlisting}[moreemph={Array}]
for (int i = 0; i < b.size(); ++i)
    cout << b[i] << " ";
\end{lstlisting}
\end{blueblock}
\uncover<2->{输出结果为:\texttt{1 2 3 0 0}}

\column{0.3\textwidth}
\begin{yellowblock}{说明}
创建对象\texttt{b}时,将执行具有形参的构造函数,其形参\texttt{l}接受初始化列表 \texttt{\{1,2,3\}},其余元素具有默认值\texttt{0}
\end{yellowblock}


\end{columns}

\end{frame}

%-----------------

%-------------------------------------
\subsection{默认模板参数}
%-------------------------------------

%-----------------

\begin{frame}[fragile]{7.2.3~默认模板参数}

\alert{函数参数}可以具有默认值,\alert{模板参数}同样也可以有默认值:

\vspace{-4mm}

\begin{columns}[t]

\column{0.65\textwidth}
\begin{blueblock}{\texttt{Array}类模板定义二}
\vspace{-1mm}\begin{lstlisting}[moreemph={Array,T,Less,F}]
template<typename T = int, size_t N = 10>
class Array {
    // 其它成员保持不变
};
\end{lstlisting}\vspace{-1mm}
\end{blueblock}
\begin{blueblock}<2->{实例化\texttt{Array}类模板二}
\begin{lstlisting}[moreemph={Array}]
Array<> a = { 'A' };
cout << a.size() << " " << a[0] << endl;
\end{lstlisting}
\end{blueblock}
\uncover<2->{输出结果为:\texttt{10 65}}

\column{0.3\textwidth}
\begin{yellowblock}{说明}
$\bullet$ \alert{类模板参数}~\texttt{T}具有默认类型\texttt{int}\\
$\bullet$ \alert{类模板参数}~\texttt{N}具有默认值\texttt{10}
\end{yellowblock}
\begin{yellowblock}<2->{说明}
$\bullet$ \texttt{a}的元素数目为默认值\texttt{10}\\
$\bullet$ \texttt{a[0]}的类型为\texttt{int},字符\texttt{'A'}转换为\texttt{65}
\end{yellowblock}

\end{columns}

\end{frame}

%-----------------

\begin{frame}[fragile]{7.2.3~默认模板参数}

为\alert{函数模板参数}提供默认值:

\vspace{-4mm}

\begin{columns}[t]

\column{0.65\textwidth}
\begin{blueblock}{\texttt{Array}类模板定义三}
\vspace{-2mm}\begin{lstlisting}[moreemph={Array,T,Less,F}]
template<typename T, size_t N = 10>
class Array {
    // 其它成员保持不变
public:
    template<typename F = Less<T>>
    void sort(F f = F());
};
\end{lstlisting}\vspace{-2mm}
\end{blueblock}
\begin{blueblock}<2->{\texttt{Less}类模板定义}
\vspace{-2mm}\begin{lstlisting}[moreemph={Less,T}]
template<typename T>
struct Less{
    bool operator()(const T &a, const T &b) {
        return a < b;
    }
};
\end{lstlisting}\vspace{-2mm}
\end{blueblock}

\column{0.3\textwidth}
\begin{yellowblock}{说明}
$\bullet$ 新增了一个成员函数模板\texttt{sort},用来对数组进行排序\\
$\bullet$ \texttt{sort}的\alert{函数模板参数}~\texttt{F}具有默认值\texttt{Less<T>}
\end{yellowblock}
\begin{yellowblock}<2->{说明}
类模板\texttt{Less<T>}具有一个模板参数\texttt{T},且只有一个函数调用运算符,该成员函数带有两个形参,用来比较两个形参的大小,返回值类型为\texttt{bool}
\end{yellowblock}

\end{columns}

\end{frame}

%-----------------

\begin{frame}[fragile]{7.2.3~默认模板参数}

和\alert{类模板参数}一样,\alert{函数模板参数}也可以有默认值:

\vspace{-4mm}

\begin{columns}[t]

\column{0.65\textwidth}
\begin{blueblock}{\texttt{Array}类模板定义三}
\vspace{-1mm}\begin{lstlisting}[moreemph={Array,T,Less,F}]
template<typename T, size_t N = 10>
class Array {
    // 其它成员保持不变
public:
    template<typename F = Less<T>>
    void sort(F f = F());
};
\end{lstlisting}\vspace{-1mm}
\end{blueblock}

\column{0.3\textwidth}
\begin{yellowblock}{说明}
$\bullet$ \texttt{sort}的\alert{函数参数}~\texttt{f}也有默认值,即\texttt{F}类的一个函数对象,代表默认比较方式为\texttt{Less}
\end{yellowblock}
\begin{greenblock}<2->{问题}
理清~\alert{函数参数}和\alert{模板参数}的概念
\end{greenblock}

\end{columns}

\end{frame}

%-----------------

